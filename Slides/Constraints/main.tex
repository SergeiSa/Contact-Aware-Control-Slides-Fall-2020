\documentclass{beamer}

\input{settings.tex}


\title{Constraints}
\subtitle{Contact-aware Control, Lecture 2}
\author{by Sergei Savin}
\centering
\date{Fall 2020}



\begin{document}
\maketitle


\begin{frame}{Content}

\begin{itemize}
\item Dynamical systems
\begin{itemize}
\item Example 1: pendulum dynamics equations
\end{itemize}
\item Constraints
\item Example 2: double oscillator
\begin{itemize}
\item parts 1-5
\end{itemize}
\item Homework
\end{itemize}

\end{frame}



\begin{frame}{Dynamical systems}
% \framesubtitle{O}
\begin{flushleft}

\emph{Dynamical system} is this course means an ODE. A general form of an ODE is:

\begin{equation}
    \dot{\bo{x}} = \bo{f}(\bo{x}, t) 
\end{equation}

Any ODE non-degenerate can be represented in this form. 

\begin{exampleblock}{Example 1}
For example, consider pendulum dynamics equations:

\begin{equation}
    \ddot{\phi} = -l sin(\phi) 
\end{equation}

Introducing a change of coordinates $\bo{x} = [\phi \ \dot{\phi}]$ we get:

\begin{equation}
    \begin{bmatrix} \dot{x}_1 \\ \dot{x}_2 \end{bmatrix} = 
    \begin{bmatrix} x_2 \\ -l sin(x_1) \end{bmatrix} 
\end{equation}

\end{exampleblock}

\end{flushleft}
\end{frame}




\begin{frame}{Constraints}
% \framesubtitle{O}
\begin{flushleft}

A \emph{constraint} is an equality that must hold for a given dynamical system, as its state evolves in time.

\bigskip

Consider a general-form dynamical system:

\[
    \dot{\bo{x}} = \bo{f}(\bo{x}, t) 
\]

A general-form algebraic constraint for this system can be written as:

\begin{equation}
    \bo{g}(\bo{x}) = 0 
\end{equation}

\bigskip

A differential constraint would have form $\bo{g}(\bo{x}, \ \dot{\bo{x}}) = 0$. Both definitions are quite obvious.

\end{flushleft}
\end{frame}


\begin{frame}{Example 2: double oscillator}
\framesubtitle{Part 1}
\begin{flushleft}

Remember the example from the previous lecture:

\begin{equation}
\begin{cases}
    m_1 \ddot x_1 = k_1 x_1 + k_2 (x_2 - x_1 - l_{12}) + f_1 (t) \\
    m_2 \ddot x_2 = -k_2 (x_2 - x_1 - l_{12}) + f_2 (t)
\end{cases}
\end{equation}

where $f_1 (t)$ and $f_2 (t)$ are external forces (we add then to make some observations later).

\bigskip

We can require that $x_1 - x_2 = 1$, (meaning the distance between two bodies remains to be equal to 1). The equation is then changed to:

\begin{equation}
\begin{cases}
    m_1 \ddot x_1 = k_1 x_1 + k_2 (x_2 - x_1 - l_{12}) + f_1 (t) \\
    m_2 \ddot x_2 = -k_2 (x_2 - x_1 - l_{12}) + f_2 (t) \\
    x_1 - x_2 = 1
\end{cases}
\end{equation}

\end{flushleft}
\end{frame}




\begin{frame}{Example 2: double oscillator}
\framesubtitle{Part 2}
\begin{flushleft}

We can differentiate the constraint twice and get the following equation:

\begin{equation}
\begin{cases}
    m_1 \ddot x_1 = k_1 x_1 + k_2 (x_2 - x_1 - l_{12}) + f_1 (t) \\
    m_2 \ddot x_2 = -k_2 (x_2 - x_1 - l_{12}) + f_2 (t) \\
    \ddot x_1 - \ddot x_2 = 0
\end{cases}
\end{equation}

which is equivalent to the original, as long as the initial condition satisfies the constraint $x_1 - x_2 = 1$. 

\bigskip

Notice, that this equation is not guaranteed to have a solution. In fact, first two equations remained unchanged, they lack \emph{force} that would ensure the constraint holds. 

\end{flushleft}
\end{frame}



\begin{frame}{Example 2: double oscillator}
\framesubtitle{Part 3}
\begin{flushleft}

Adding a new force that acts on the first equation:

\begin{equation}
\begin{cases}
    m_1 \ddot x_1 = k_1 x_1 + k_2 (x_2 - x_1 - l_{12}) + f_1 (t) + \lambda \\
    m_2 \ddot x_2 = -k_2 (x_2 - x_1 - l_{12}) + f_2 (t) \\
    \ddot x_1 - \ddot x_2 = 0
\end{cases}
\end{equation}

It is now \emph{always possible} to find such $\lambda$ that the constraint holds.

\end{flushleft}
\end{frame}



\begin{frame}{Example 2: double oscillator}
\framesubtitle{Part 4}
\begin{flushleft}

Let us consider the case when $k_1 = k_2 = 0$, $m_1 = m_2 = 1$. Then we have:

\begin{equation}
\begin{cases}
    \ddot x_1 = f_1 (t) + \lambda \\
    \ddot x_2 = f_2 (t) \\
    \ddot x_1 - \ddot x_2 = 0
\end{cases}
\end{equation}

Notice that $\ddot x_1 = \ddot x_2 = f_2 (t)$ and $\lambda = f_2 (t) - f_1 (t)$. The work produced by this force is $\int \dot x_1 (f_2 (t) - f_1 (t)) dt$, and  in general does not have to be equal to 0.

\end{flushleft}
\end{frame}




\begin{frame}{Example 2: double oscillator}
\framesubtitle{Part 5}
\begin{flushleft}

Now consider that the \emph{reaction force} is applied to both equations and in this way:

\begin{equation}
\label{Example1}
\begin{cases}
    \ddot x_1 = f_1 (t) - \lambda \\
    \ddot x_2 = f_2 (t) + \lambda\\
    \ddot x_1 - \ddot x_2 = 0
\end{cases}
\end{equation}

Then the reaction force is $\lambda = (f_2 (t) - f_1 (t)) / 2$ (Prove it!). The work produced by this force is $\int -\dot x_1 \frac{f_2 (t) - f_1 (t)}{2} + \dot x_2 \frac{f_2 (t) - f_1 (t)}{2} dt$, which is equal to zero as long as the constraint holds, since the constraint implies that $\dot x_1 - \dot x_2 = 0$.

\bigskip

Thus, there are some constraints that produce no mechanical work.

\end{flushleft}
\end{frame}



\begin{frame}{Homework}
% \framesubtitle{Parameter estimation}
\begin{flushleft}

(I) Prove that in \eqref{Example1} the value of the reaction force is $\lambda = (f_2 (t) - f_1 (t)) / 2$.

\bigskip

(II) For the system:

\begin{equation}
\begin{cases}
    \ddot x_1 = f_1 (t) \\
    \ddot x_2 = f_2 (t) \\
    \ddot x_3 = 1
\end{cases}
\end{equation}

add reaction force $\lambda$ that enforces constraint $\ddot x_1 + \ddot x_2 - \ddot x_3 = 0$, which would produce no work.

\end{flushleft}
\end{frame}




\begin{frame}
\centerline{Lecture slides are available via Moodle.}
\bigskip
\centerline{You can help improve these slides at:}
\centerline{\href{https://github.com/SergeiSa/Contact-Aware-Control-Slides-Fall-2020}{github.com/SergeiSa/Contact-Aware-Control-Slides-Fall-2020}}
\bigskip
\centerline{Check Moodle for additional links, videos, textbook suggestions.}
\end{frame}

\end{document}
