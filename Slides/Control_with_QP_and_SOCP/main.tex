\documentclass{beamer}

\input{settings.tex}


\title{Control with QP and SOCP}
\subtitle{Contact-aware Control, Lecture 8}
\author{by Sergei Savin}
\centering
\date{Fall 2020}



\begin{document}
\maketitle


\begin{frame}{Content}

\begin{itemize}
\item Quadratic programming
\begin{itemize}
\item Special case with an analytic solution
\item Example
\item Example - solution using quadprog
\item Example - solution using CVX
\item Application
\item What it can’t do
\end{itemize}
\item Second-order cone programming
\item Friction cone and SOCP
\begin{itemize}
\item Formulation
\item Application
\end{itemize}
\item Read more
\end{itemize}

\end{frame}



\begin{frame}{Quadratic programming}
% \framesubtitle{Part 1}
\begin{flushleft}

There are special cases of optimization problems that can be solved numerically very efficiently. One of those is \emph{quadratic programming}.

\bigskip

General form of a quadratic program (QP) is given below:

%
\begin{equation}
\begin{aligned}
& \underset{\mathbf{x}}{\text{minimize}}
& & \mathbf{x}^\top \mathbf{H} \mathbf{x} + \mathbf{f}^\top\mathbf{x}, \\
& \text{subject to}
& & \begin{cases}
    \mathbf{A}\mathbf{x} \leq \mathbf{b}, \\
    \mathbf{C}\mathbf{x} = \mathbf{d}.
    \end{cases}
\end{aligned}
\end{equation}

where $\mathbf{H}$ is positive-definite and $\mathbf{A}\mathbf{x} \leq \mathbf{b}$ describe a \emph{convex region}.

\end{flushleft}
\end{frame}



\begin{frame}{Quadratic programming}
\framesubtitle{Special case with an analytic solution}
\begin{flushleft}

If there are no inequality constraints: 

\begin{equation}
\begin{aligned}
& \underset{\mathbf{x}}{\text{minimize}}
& & \mathbf{x}^\top \mathbf{H} \mathbf{x} + \mathbf{c}^\top\mathbf{x}, \\
& \text{subject to}
& & \mathbf{A} \mathbf{x} = \mathbf{b}.
\end{aligned}
\end{equation}

a quadratic program can be solved analytically.

\end{flushleft}
\end{frame}



\begin{frame}{Quadratic programming}
\framesubtitle{Example}
\begin{flushleft}

We have the following problem: find such $\mathbf x$ that minimizes $\mathbf x^\top \mathbf M \mathbf x$, while $\mathbf C \mathbf x = \mathbf y$. In other words:

\begin{equation} \label{eq:Example1:1}
\begin{aligned}
& \underset{\mathbf  x}{\text{minimize}}
& & \mathbf x^\top \mathbf M \mathbf x, \\
& \text{subject to}
& & \mathbf C \mathbf x = \mathbf y.
\end{aligned}
\end{equation}

More concrete:

\begin{equation} \label{eq:Example1:2}
\begin{aligned}
& \underset{\mathbf  x}{\text{minimize}}
& & \begin{bmatrix} x_1 & x_2 & x_3 \end{bmatrix}
\begin{bmatrix}  
   1 & 0 & 1 \\
   0 & 5 & 0 \\
   1 & 0 & 3 \\
   \end{bmatrix}
\begin{bmatrix} x_1 \\ x_2 \\ x_3 \end{bmatrix}, \\
& \text{subject to}
& & \begin{bmatrix} 1 & 7 & 2 \end{bmatrix}
\begin{bmatrix} x_1 \\ x_2 \\ x_3 \end{bmatrix}
= 1.
\end{aligned}
\end{equation}

\end{flushleft}
\end{frame}




\begin{frame}{Quadratic programming}
\framesubtitle{Example - solution using quadprog}
\begin{flushleft}

We can use a dedicated solver for this class of problems - \texttt{quadprog} provided by MATLAB. Here is the solution:

\begin{lstlisting}[language=Matlab]
M = [1 0 1; 0 5 0; 1 0 3]; 
C = [1 7 2]; 
y = 1;

x = quadprog(M,[],[],[],C,y)
\end{lstlisting}

Average solution time is \textbf{0.56} ms

\end{flushleft}
\end{frame}




\begin{frame}{Quadratic programming}
\framesubtitle{Example - solution using CVX}
\begin{flushleft}

Alternatively we can invoke one of the most powerful convex optimization tools with a user-friendly coding style - CVX:

\begin{lstlisting}[language=Matlab]
M = [1 0 1; 0 5 0; 1 0 3];
C = [1 7 2];
y = 1;

cvx_begin
variables x(3);
minimize( x' * M * x );
subject to
    C*x == y;
cvx_end
\end{lstlisting}

\href{http://cvxr.com/cvx/}{Official web page of the CVX package}

\end{flushleft}
\end{frame}






\begin{frame}{Quadratic programming}
\framesubtitle{Application}
\begin{flushleft}

Remember CTC control from the lecture on error dynamics:

\begin{equation}
\begin{cases}
    \bo{v} = \bo{H}(\ddot{\bo{q}}^* + \bo{K}_d\dot{\bo{e}} + \bo{K}_p\bo{e}) + \bo{c} \\
    \bo{T}\bo{u} + \bo{F}^\top \lambda = \bo{v}
\end{cases}
\end{equation}

Assume that $\lambda = [\lambda_1^\top, \ ...  \lambda_m^\top]^\top \in \R^{3m}$, $\lambda_i \in \R^{3}$ and represents a $m$ individual unilateral contact with normal unit-vectors to the contact surface $\bo{n}_i$, and no friction. Then we can write the control law as a solution to the quadratic program:

\begin{equation}
\begin{aligned}
& \underset{\bo{u}, \bo{v}, \lambda_i}{\text{minimize}}
& & \mathbf{u}^\top \mathbf{u}, \\
& \text{subject to}
& & \begin{cases}
    \bo{H}(\ddot{\bo{q}}^* + \bo{K}_d\dot{\bo{e}} + \bo{K}_p\bo{e}) + \bo{c} = \bo{v} \\
    \bo{T}\bo{u} + \sum\limits_{i = 1}^m \bo{F}_i^\top \lambda_i = \bo{v} \\\
    -\bo{n}_i^\top \lambda_i \leq 0
\end{cases}
\end{aligned}
\end{equation}

\end{flushleft}
\end{frame}



\begin{frame}{Quadratic programming}
\framesubtitle{What it can't do}
\begin{flushleft}

Remember friction cone constraints:

\begin{equation}
\label{eq:friction_cone_2}
    || [\tau_1 \ \tau_2]^\top \bo{f} || \leq \mu \bo{n}^\top \bo{f}
\end{equation}

These can't be directly represented in a quadratic program.

\end{flushleft}
\end{frame}



\begin{frame}{Second-order cone programming}
\framesubtitle{General form}
\begin{flushleft}

The general form of a Second-order cone program (SOCP) is:

%
\begin{equation}
\begin{aligned}
& \underset{\mathbf{x}}{\text{minimize}}
& & \mathbf{f}^\top\mathbf{x}, \\
& \text{subject to}
& & \begin{cases}
    ||\mathbf{A}_i\mathbf{x} + \mathbf{b}_i||_2 \leq 
     \mathbf{c}_i^\top \mathbf{x} + d_i, \\
    \mathbf{F}\mathbf{x} = g.
    \end{cases}
\end{aligned}
\end{equation}

QP are a subset of SOCP.
 
\end{flushleft}
\end{frame}



\begin{frame}{Friction cone and SOCP}
% \framesubtitle{O}
\begin{flushleft}

We already saw that a no-friction contact can be represented as a QP. Next we will try to show that friction cone (a more general model) can be represented as a SOPC (a more general optimization problem type).

\begin{figure}
    \centering
    \includegraphics[width= 0.85\linewidth]{fig2.png}
    % \caption{}
    \label{fig:contact}
\end{figure}

\scriptsize{Picture is from \emph{Sancho-Bru, J.L., P\'{e}rez-Gonz\'{a}lez, A., Mora, M.C., L\'{o}n, B.E., Vergara, M., Iserte, J.L., Rodriguez-Cervantes, P.J. and Morales, A., 2011. Towards a realistic and self-contained biomechanical model of the hand. In Theoretical biomechanics. IntechOpen.}}

\end{flushleft}
\end{frame}



\begin{frame}{Friction cone and SOCP}
\framesubtitle{Formulation}
\begin{flushleft}

Coming back to the the friction cone general view:

\begin{equation}
    || [\tau_1 \ \tau_2]^\top \bo{f} || \leq \mu \bo{n}^\top \bo{f}
\end{equation}

This is has the form of a conic constraint, which is admissible for SOCP:

\begin{equation}
    ||\mathbf{A}_i\mathbf{x} + \mathbf{b}_i||_2 \leq 
     \mathbf{c}_i^\top \mathbf{x} + d_i
\end{equation}


\end{flushleft}
\end{frame}



\begin{frame}{Friction cone and SOCP}
\framesubtitle{Application}
\begin{flushleft}

Consider this problem again:

\begin{equation}
\begin{cases}
    \bo{v} = \bo{H}(\ddot{\bo{q}}^* + \bo{K}_d\dot{\bo{e}} + \bo{K}_p\bo{e}) + \bo{c} \\
    \bo{T}\bo{u} + \bo{F}^\top \lambda = \bo{v}
\end{cases}
\end{equation}

But this time we require that:

\begin{equation}
    || [\tau_{i,1} \ \tau_{i,2}]^\top \lambda_i || \leq \mu \bo{n}_i^\top \lambda_i
\end{equation}


\end{flushleft}
\end{frame}



\begin{frame}{Friction cone and SOCP}
\framesubtitle{Application, part 2}
\begin{flushleft}

Solution can take the form of a second-order cone program:

\begin{equation}
\begin{aligned}
& \underset{\bo{u}, \bo{v}, \lambda_i}{\text{minimize}}
& & \mathbf{u}^\top \mathbf{u}, \\
& \text{subject to}
& & \begin{cases}
    \bo{H}(\ddot{\bo{q}}^* + \bo{K}_d\dot{\bo{e}} + \bo{K}_p\bo{e}) + \bo{c} = \bo{v} \\
    \bo{T}\bo{u} + \sum\limits_{i = 1}^m \bo{F}_i^\top \lambda_i = \bo{v} \\\
    || [\tau_{i,1} \ \tau_{i,2}]^\top \lambda_i || \leq \mu \bo{n}_i^\top \lambda_i
\end{cases}
\end{aligned}
\end{equation}


\end{flushleft}
\end{frame}







\begin{frame}{Read more}
% \framesubtitle{Parameter estimation}
\begin{flushleft}

You can read and watch more at:

\begin{itemize}
    \item \href{http://web.cvxr.com/cvx/doc/}{CVX user guide}
    \item \href{https://www.mathworks.com/help/optim/ug/quadprog.html}{quadprog - MATLAB documentation}
    \item \href{https://youtu.be/bbyF89OnpBo}{Computational Intelligence 2020, Lecture 5 (Least Squares and Quadratic Programming)}
    \item \href{https://youtu.be/sVkUht9-py0}{Computational Intelligence 2020, Lecture 6 (Domain, Convex Domains)}
    \item \href{https://youtu.be/4FboGNcsQhU}{Computational Intelligence 2020, Lecture 7 (Linear inequality representation of convex domains)}
    \item \href{https://youtu.be/c4qroDnvDak}{Computational Intelligence 2020, Lecture 8 (QCQP, SOCP)}
\end{itemize}



\end{flushleft}
\end{frame}




\begin{frame}
\centerline{Lecture slides are available via Moodle.}
\bigskip
\centerline{You can help improve these slides at:}
\centerline{\href{https://github.com/SergeiSa/Contact-Aware-Control-Slides-Fall-2020}{github.com/SergeiSa/Contact-Aware-Control-Slides-Fall-2020}}
\bigskip
\centerline{Check Moodle for additional links, videos, textbook suggestions.}
\end{frame}

\end{document}
